\documentclass[nogrid]{MBE}%

\usepackage{url}

\jshort{mst}

\volname{}

\jvolume{0}

\jvol{}

\jissue{0}

\pubyear{2019}

\mstype{Article}

\artid{012}

\begin{document}


\title{Mode and tempo of promoter evolution in the \textit{Paramecium aurelia} species complex}


\author[Raborn et al.]{R. Taylor \surname{Raborn},$^{\ast,1,2}$ Timothy Licknack,$^{1,2}$ Wanfeng Guo,$^{1,2}$ Shannon N. Snyder,$^{1,2}$ and Michael Lynch$^{1,2}$}

\address{$^{1}$Biodesign Center for Mechanisms of Evolution\\
$^{2}$School of Life Sciences\\
Arizona State University, Office C445 797 E. Tyler Street, Tempe, AZ 85281}


%\history{Received xx xxxxx 2019; reviews returned xx xxxx 2019; accepted xx xxxxx 2019}

\coresp{E-mail: rtraborn@asu.edu}

<<<<<<< HEAD
\abstract{Goes here
}

\keyword{cis-regulatory regions, duplicate genes, paralogous genes, Paramecium, promoter evolution, TSS profiling.}

\maketitle


\section{{Introduction}\label{sec:Intro}}

Genomic analysis across diverse eukaryotes has drawn attention to the important role of \textit{cis}-regulatory sequences in the evolution of gene expression \citep{Wittkopp:2008ki, Wittkopp:2011bc, Siepel:2014hd}. However, the \textit{cis}-regulatory sequence differences that accompany--or indeed underpin--species divergence remain largely unknown. A likely explanation for this is the difficulty in predicting \textit{cis}-regulatory regions from genomic sequence alone. Accurate estimation of promoter positions requires empirical, usually functional genomic evidence. At present, the best high-throughput approach to identify promoters at genome-scale is Transcription Start Site (TSS) profiling, which includes CAGE and RAMPAGE, among others. While TSS profiling methods differ methadologically, these methods capture the 5`-ends of capped mRNAs, sequence their corresponding cDNAs and align the reads to the genome to identify the TSSs present within a given transcriptome. Clustering of gene-adjacent TSSs defines a promoter, or transcription start region (TSR). 


\section{Demographic structure}

Lorem ipsum dolor sit amet, consectetur adipiscing elit, sed do eiusmod tempor incididunt ut labore et dolore magna aliqua. Ut enim ad minim veniam, quis nostrud exercitation ullamco laboris nisi ut aliquip ex ea commodo consequat. Duis aute irure dolor in reprehenderit in voluptate velit esse cillum dolore eu fugiat nulla pariatur. Excepteur sint occaecat cupidatat non proident, sunt in culpa qui officia deserunt mollit anim id est laborum.

\subsection{Subsection 1}

Lorem ipsum dolor sit amet, consectetur adipiscing elit, sed do eiusmod tempor incididunt ut labore et dolore magna aliqua. Ut enim ad minim veniam, quis nostrud exercitation ullamco laboris nisi ut aliquip ex ea commodo consequat. Duis aute irure dolor in reprehenderit in voluptate velit esse cillum dolore eu fugiat nulla pariatur. Excepteur sint occaecat cupidatat non proident, sunt in culpa qui officia deserunt mollit anim id est laborum.


\subsubsection{Subsection 2}

Lorem ipsum dolor sit amet, consectetur adipiscing elit, sed do eiusmod tempor incididunt ut labore et dolore magna aliqua. Ut enim ad minim veniam, quis nostrud exercitation ullamco laboris nisi ut aliquip ex ea commodo consequat. Duis aute irure dolor in reprehenderit in voluptate velit esse cillum dolore eu fugiat nulla pariatur. Excepteur sint occaecat cupidatat non proident, sunt in culpa qui officia deserunt mollit anim id est laborum.


\paragraph{Paragraph header} 

Lorem ipsum dolor sit amet, consectetur adipiscing elit, sed do eiusmod tempor incididunt ut labore et dolore magna aliqua. Ut enim ad minim veniam, quis nostrud exercitation ullamco laboris nisi ut aliquip ex ea commodo consequat. Duis aute irure dolor in reprehenderit in voluptate velit esse cillum dolore eu fugiat nulla pariatur. Excepteur sint occaecat cupidatat non proident, sunt in culpa qui officia deserunt mollit anim id est laborum.

\section{{Methods}\label{sec:Methods}}

Lorem ipsum dolor sit amet, consectetur adipiscing elit, sed do eiusmod tempor incididunt ut labore et dolore magna aliqua. Ut enim ad minim veniam, quis nostrud exercitation ullamco laboris nisi ut aliquip ex ea commodo consequat. Duis aute irure dolor in reprehenderit in voluptate velit esse cillum dolore eu fugiat nulla pariatur. Excepteur sint occaecat cupidatat non proident, sunt in culpa qui officia deserunt mollit anim id est laborum.

\section{{Results}\label{sec:Results}}

Lorem ipsum dolor sit amet, consectetur adipiscing elit, sed do eiusmod tempor incididunt ut labore et dolore magna aliqua. Ut enim ad minim veniam, quis nostrud exercitation ullamco laboris nisi ut aliquip ex ea commodo consequat. Duis aute irure dolor in reprehenderit in voluptate velit esse cillum dolore eu fugiat nulla pariatur. Excepteur sint occaecat cupidatat non proident, sunt in culpa qui officia deserunt mollit anim id est laborum.

\section{{Discussion}\label{sec:Discussion}}

Lorem ipsum dolor sit amet, consectetur adipiscing elit, sed do eiusmod tempor incididunt ut labore et dolore magna aliqua. Ut enim ad minim veniam, quis nostrud exercitation ullamco laboris nisi ut aliquip ex ea commodo consequat. Duis aute irure dolor in reprehenderit in voluptate velit esse cillum dolore eu fugiat nulla pariatur. Excepteur sint occaecat cupidatat non proident, sunt in culpa qui officia deserunt mollit anim id est laborum.

\begin{arabiclist}
\item Item 1

\item Item 2

\item Item 3
\end{arabiclist}

\begin{itemize}
\item Consider a fall in population induced by a decline in the number of births in the economy,
taking as given mortality and migration.

\item It is well known that a lower population growth raises the capital--labor ratio in the Solow--Swan
growth model.

\item The same property holds in Diamond's (1965) overlapping generations model, and it enhances welfare
as long as the economy is dynamically efficient; i.e., when the interest rate exceeds the
population growth rate.
\end{itemize}
 A similar trend is observed in the United
States and advanced European countries (Gustafsson and Kalwij, 2006), and also in Canada,
Australia, and New Zealand (Sardon, 2006). Interestingly, as pointed out by Bongaarts and Feeney
(1998), even when the cohort's lifetime fertility rate (the number of children a mother has in her
lifetime) does not fall, the delayed childbearing alone leads to a decline in the number of
childbirths, measured by the total period fertility rates (TPFRs). %Ogawa and Retherford (1993),
%Kohler et al. (2002), and Sobotka (2004) confirmed that, to a certain extent, the delay of
%marriage and motherhood is responsible for the observed period fertility rate decline (now known
%as the `tempo effect').


\section{Model\label{sec:Model}}

\subsection{Demographic structure}



i.e.:
\begin{equation}
\lambda_{t}=\left\{
\begin{array}
[c]{cc}%
0, & t<0,\\
\lambda, & t\geq0.
\end{array}
\right.  \label{eq:lambda}%
\end{equation}


\begin{figure}[t]
\begin{center}
\includegraphics[height=0.21\textheight]{flrf1.eps}
\end{center}
\caption{Fluctuations in Cohort Size $N_{t}$ over Generations.}%
\label{fig:popdynamics}%
\end{figure}

%\TWOfig{\includegraphics[width=10pc]{flrf1.eps}}{Fluctuations in
%Cohort Size $N_{t}$ over
%Generations.\label{fig:popdynamics}}{\includegraphics[width=10pc]{flrf1.eps}}{Dynamics
%of Labor Force $L_{t}$.\newline
%\phantom{\hskip5pc}\label{fig:labordynamics}}

where $C$ is a constant term defined as $C\equiv\beta\log\beta-(1+\beta
)\log(1+\beta)+\beta\log A\alpha+(1+\beta)\log A\left(  1-\alpha\right)  $.
Similarly, long-term welfare in the benchmark economy ($\lambda=0$) can be
written as:
\begin{equation}
U^{\ast}=\left(  1+\beta\right)  \log[A\alpha\left(  k^{\ast}\right)
^{2\alpha-1}+\left(  k^{\ast}\right)  ^{\alpha}]-\beta\left(  1-\alpha\right)
\log k^{\ast}+C. \label{eq:U_benchmark}%
\end{equation}

\begin{table}[!t]%1
\tableparts{\caption{SH test results on nuclear and mitochondrial phylogenetic trees}\label{tab1}}
{\begin{tabular*}{\columnwidth}{@{\extracolsep{\fill}}lld{6,0}d{6,0}@{}}\toprule Sequence data &
\mcc{Tree} & \mcc{$-\ln~L$} & \mcc{SH test $P$-value} \\\colrule mtDNA& mtDNA& -109219.5& 0.5 \\
[0.1pt]
mtDNA& Nuclear& -61720.8& \mcc{\hspace*{5pt}$<{0.00001}$} \\
Nuclear& mtDNA& -113033.1& \mcc{\hspace*{5pt}$<{0.00001}$} \\
Nuclear& Nuclear& -60699.9& 0.5 \\\botrule
\end{tabular*}}
{}
\end{table}

%
\section{Supplementary Material}
Supplementary tables S1?S7 and figures S1?S11 are available  at Molecular Biology and Evolution
online (http://www.mbe.oxfordjournals.org/).

\section{Acknowledgments}

The authors gratefully acknowledge the help of Robert Policastro and Gabriel Zentner at Indiana University Bloomington for technical assistance and use of laboratory facilities.


\bibliographystyle{natbib}%%%%natbib.sty
\bibliography{refs}%%%refs.bib

\end{document}
